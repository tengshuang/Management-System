\chapter{引言}

得益于互联网的发展,现在越来越多的行政管理系统采取了无纸化操作的模式。借由互联网的便利性、实时性、安全性和可靠性作为保证,越来越多的工作可以在互联网上方便地进行。以清华大学的互联网系统为例,清华大学已经拥有了完整成体系的清华大学信息门户系统。辅以清华大学网络学堂,清华大学在线图书馆,清华学生生活,清华大学学业门户,清华大学在线选课系统,清华大学的在线系统已经成相当规模,为师生的学习生活带来了便利,使得学生们可以足不出户地处理各项学习生活服务。就在最近两年,清华大学的自助注册系统上线,更进一步简化了学生注册的步骤。利用互联网技术进行自助服务已经成为了清华大学各项工作改革的主流趋势。

随着Web2.0乃至Web3.0的发展,网页不再是单向向用户传播信息的工具。借由一套基于浏览器的管理系统,用户现在可以通过浏览器向服务器发送控制和命令,以达到管理的目的。这伴随着浏览器的功能逐渐完善,以及得益于W3C组织对浏览器进行标准化的工作,区别于传统的基于客户端的C/S架构的基于浏览器的B/S架构正变得越来越流行。基于B/S架构的行政系统可以脱离一个独立的客户端,直接以网页的形式在网页浏览器中进行原来需要一个独立的客户端才能进行的工作。

另一方面,随着智能手机的不断普及,移动平台的应用也越来越成熟。随着微信开放开发环境的成熟,比起自行开发一个应用,越来越多的企业和学校服务开始倾向于选择微信微平台作为提供服务的平台。这保证了这些服务在提供服务的同时,保证了用户在使用服务时同时能享受到微信方便成熟的社交功能。统一的界面也免去了用户学习应用使用方法的过程,也免去了二次验证和用户账户管理的繁琐,因此出现了这种新兴的C/S架构,即将微信作为自己的C/S,与微信内嵌的浏览器结合,形成B/S和C/S混合架构来提供更完整的服务。

目前,清华大学校内还有许多行政进程没有进行无纸化互联网转变。清华大学校内的工会,暂时还在使用传统人工的方式进行活动发布、人员管理和签到记录。使用这种传统方式进行的方法,既影响了工会工作的效率,又有可能会导致资料丢失和错误,为未来的工作带来不必要的麻烦。

实际上,清华大学工会可以采用一套基于互联网的行政管理系统来统一管理这一系列的流程。本文就将以清华大学的工会管理系统为例,呈现一套完整的基于B/S和C/S混合架构的清华大学工会管理系统的详细架构,具体实现,并和现有的行政系统进行对比,对结论进行阐释。
