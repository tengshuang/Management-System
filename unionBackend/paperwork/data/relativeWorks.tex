\chapter{相关工作}

目前已经有相当一部分的互联网行政管理系统的实现\cite{oyj2006}\cite{zw2015}\cite{ysy2000}\cite{wq2010}\cite{sj2004}\cite{hk2013},其中不乏与本文中同样属于工会管理系统的实现案例\cite{hzj2007}\cite{whx2006}\cite{dxl2007}\cite{lzq2010}。然而这些工作\cite{wq2010}\cite{hzj2007}在鼓吹B/S相对于C/S的优越性的时候,却忽略了移动设备上微平台(例如微信)对C/S架构发展的贡献。同时,这些系统往往更多在考虑和介绍自己的针对自己需求的实现,系统存在局限性,无法将设计模式向其他的系统进行外推。更严重的是,这些系统对网络传输的安全性鲜有考虑。对于用户的数据来说,这样的行为是不负责任并且是十分危险的。

也有许多的工会系统是基于微信移动平台进行开发的。其中不乏有将微信开发平台运用在工会系统上的实现\cite{lih2015}。然而这项工作却缺少对技术细节的详细叙述,并且仅仅使用了微信平台作为唯一的信息发布渠道。实际上,碍于移动设备的输入方式受到限制,在微信上进行含有具体内容的信息的发布效率并不高。因此单纯的微信平台会导致信息发布存在障碍,这项工作需要一个基于浏览器的控制端才能高效地运作。其他一些在微信开发平台上的实现\cite{fanfl2013}\cite{qinjj2015}则与行政系统的需求差距比较远,无法适用于本文中系统使用的环境当中,但是其中的使用微信开发的思想却是值得借鉴的。

参考这些现有工作的实现,并结合这些工作中的优点,对其中的缺点进行分析之后,我们可以从中抽象出一套适用于清华大学工会管理系统的系统架构。
